


\subsection{Definition of Time Series}

A \textbf{Time Series} is a collection of data observations made sequentially over time.\\

\textbf{Examples of Time Series (from various scientific fields):}
\begin{itemize}
    \item \textbf{Economics}: Examples include unemployment rates, Gross National Product (GNP), gold prices, and exchange rates over time.
    \item \textbf{Finance}: Examples include stock returns and other financial metrics.
    \item \textbf{Geology}: Examples include temperature records, rainfall amounts, snow mass, and similar environmental measures.
    \item And many other fields.
\end{itemize}

\subsection{Types of Time Series (Terminology):}
\begin{enumerate}
    \item \textbf{Continuous Time Series}: These are observed at every point in time.
    \begin{itemize}
        \item Examples: humidity measurements, etc.
    \end{itemize}
    \item \textbf{Discrete Time Series}: These are observed at specific intervals.
    \begin{itemize}
        \item Examples: daily stock returns, quarterly unemployment figures, yearly avalanche deaths.
    \end{itemize}
\end{enumerate}

\subsection{Objectives in Time Series Analysis:}
\begin{enumerate}
    \item \textbf{Description}
    \begin{itemize}
        \item Begin with a time plot, where time is on the x-axis and the observed series value is on the y-axis.
        \item Look for features such as trends, seasonal variations, cyclical variations, regime shifts, and outliers.
    \end{itemize}
    \item \textbf{Explanation}
    \begin{itemize}
        \item More applicable to multivariate time series.
        \item For example, one might examine the relationship between unemployment and GDP.
        \item Methods to explain one time series using another include linear regression, nonlinear regression, and linear systems analysis.
    \end{itemize}
    \item \textbf{Prediction}
    \begin{itemize}
        \item Given a time series with values observed up to the present, predict the most likely future values.
        \item Examples: weather forecasting, GDP forecasting.
    \end{itemize}
    \item \textbf{Control}
    \begin{itemize}
        \item In many applications, it is possible to intervene and change the course of a time series.
    \end{itemize}
    \item \textbf{Statistical Inference}
    \begin{itemize}
        \item Involves conducting hypothesis tests or creating confidence intervals about parameters or properties of a time series.
        \item For example, determining if there is statistical evidence to conclude that a given time series contains a linear (non-zero) trend.
    \end{itemize}
\end{enumerate}
