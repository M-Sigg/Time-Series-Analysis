\textbf{Building block time series processes.}\\


The simplest time series is a independent and identically distributed (iid.) series: $X_t = \varepsilon_t$, where $\varepsilon_t$ are iid.\\

We can build a more complicated process assuming assuming access to $\varepsilon_t$:

\begin{align*}
    X_t&= \varepsilon_t + 0.9\varepsilon_{t-1} &&\text{(Moving Average)} \\
    X_t&= 0.9X_{t-1} + \varepsilon_t &&\text{(Autoregressive Process)} \\
    X_t&= \pm \sqrt{|-0.8X_{t-1} + \varepsilon_t|} &&\text{(Nonlinear Process)}
\end{align*}

\subsection{Definitions}

A \textbf{stochastic process} is a collection of random variables that are ordered (indexed) in time.

\begin{itemize}
    \item[]
    \begin{itemize}
        \item Note: they may be discrete or continuous
    \end{itemize}
    \item For each $t$ there is a random variable $\Romanbar{X}_t$, and the collection can simply be called $\Romanbar{X}$.
    \item Usually "realizations" of $\Romanbar{X}_t$ are labeled $X_t$
    \item[$\rightarrow$] You can think of $\Romanbar{X}$ as a multivariate random variable, where components are ordered
\end{itemize}

Let $\Romanbar{X}$ be a stochastic process.\\

$\Romanbar{X}$ is \textbf{\underline{strictly stationary}} if the distribution (joint) of $(\Romanbar{X}_t, \Romanbar{X}_{t_2}, ..., \Romanbar{X}_{t_k} $ is invariant to shifts. (Equal to distribution of $(\Romanbar{X}_{t_1+h}, \Romanbar{X}_{t_2+h}, ..., \Romanbar{X}_{t_k+h} $)\\

Strict stationarity is difficult to satisfy and verify. \\

$\Romanbar{X}$ is \textbf{\underline{mean stationary}} if the mean function $\mu_t = E[\Romanbar{X}_t]$ is constant and does not depend on $t$.\\


$\Romanbar{X}$ is \textbf{\underline{covariance stationary}} if:
\begin{itemize}
    \item The mean function $\mu_t=\mathbb{E}[\Romanbar{X}_t]$ is constant (doesn't depend on t)
    \item The variance function $\sigma^2_t=Var(\Romanbar{X}_t)$ is constant
    \item The autocovariance function $\gamma(t,s)=Cov(\Romanbar{X}_t,\Romanbar{X}_s)$ depends only on $t-s$
\end{itemize}

Let $\tau=t-s$ $\Rightarrow \gamma(\tau) = Cov(\Romanbar{X}_t, \Romanbar{X}_s)$ only makes sense if $Cov(\Romanbar{X}_t,\Romanbar{X}_s)=Cov(\Romanbar{X}_{t+1},\Romanbar{X}_{s+1})$, etc.\\

$\Romanbar{X}$ is \textbf{\underline{weakly stationary}} (also known as \textbf{\underline{second-order stationarity}}) if it satisfies the conditions of both mean stationarity and covariance stationarity.

\subsection{The Random Walk}

\[\Romanbar{X}_t=\Romanbar{X}_{t-1}+\varepsilon_t, \quad  \varepsilon_t \text{ iid.}\]
Note: Random walk on integer lattice occurs when $\varepsilon_t \in \{-1,1\}$ 50/50 and sometimes $\Romanbar{X}_0=0$  \\

\textbf{Mean function}: (for the case $\Romanbar{X}_0=0$)
\begin{align*}
    \mu_t=\mathbb{E}[\Romanbar{X}_t] &= \mathbb{E}[\Romanbar{X}_{t-1} + \varepsilon_t]\\
    &=\mathbb{E}[\Romanbar{X}_{t-2}+\varepsilon_{t-1}+\varepsilon_t] \\
    & \vdots\\
    &=\mathbb{E}[\varepsilon_1+...+\varepsilon_t] \\
    &=t * \mathbb{E}[\varepsilon_1]
\end{align*}
\begin{itemize}
    \item If $\mathbb{E}[\varepsilon_1=0]$ (without drift), then $\mu_t=0$, which does not depend on $t$.
    \item If $\mathbb{E}[\varepsilon_t\neq 0]$, this is called a random walk with drift
\end{itemize}

\textbf{Variance function:}
\begin{align*}
    \sigma^2=var(\Romanbar{X}_t)&=var(\Romanbar{X}_{t-1} + \varepsilon_t)\\
    &=var(\Romanbar{X}_{t-2} + \varepsilon_{t-1} + \varepsilon_t) \\
    &\vdots \\
    &=var(\varepsilon_1+...+\varepsilon_t)\\
    &=var(\varepsilon_1)+...+var(\varepsilon_t)\\
    &=t * var(\varepsilon_t)
\end{align*}
$\sigma^2$ depends on $t$ if $Var(\varepsilon_1)\neq 0 \Rightarrow$ not covariance stationary \\

A popular simple model for stock prices is that $\log(price_t)$ is a random walk.

\subsection{The Backshift Operator $B$}

Let $\Romanbar{V}$ be a stochastic process. Then $B\Romanbar{V}$ is another stochastic process with \[(B\Romanbar{V})_t=\Romanbar{V}_{t-1}\]

\begin{itemize}
    \item More common notation: $B\Romanbar{V}_t=\Romanbar{V}_{t-1}$
    \item Also called "Lag" operator 
    \item $B^2\Romanbar{V}_t=B\Romanbar{V}_{t-1}=\Romanbar{V}_{t-2}$, \quad $B^k\Romanbar{V}_t=\Romanbar{V}_{t-k}$
    \begin{itemize}
        \item[] Note: Book uses $\nabla$ to be difference operator 
        \[ \nabla=1-B \text{ because: } \nabla V_t = V_t-V_{t-1}=V_t-BV_t=(1-B)V_t\]
    \end{itemize}
\end{itemize}

\textbf{Building a random walk with the backshift operator:}
\begin{align*}
    \Romanbar{X}_t&=\Romanbar{X}_{t-1}+\varepsilon_t\\
    \Rightarrow \Romanbar{X}_t-\Romanbar{X}_{t-1}=\varepsilon_t\\
    \Rightarrow(1-B)\Romanbar{X}_t=\varepsilon_t
\end{align*}

\textbf{Note:} "$1-B$" is not invertible; thus division by $1-B$ is not possible

\subsection{Moving Averages}

\begin{align*}
  \Romanbar{X}_t&=\sum_{i=0}^q \theta_i\varepsilon_{t-i} &&\text{is called a $q$-order moving average. Also called "$MA(q)$"} 
\end{align*}
\begin{align*}
  &= \underbrace{(\theta_0+\theta_1B+...+\theta_qB^q)}_{\text{polynomial in $B$ with coefficients $\theta_0...\theta_q$, real numbers usually}}\varepsilon_t
\end{align*}
\begin{align*}
  \Romanbar{X}_t&=p(B)\varepsilon_t &&\text{usually rescale $\varepsilon_t$ so that $\theta_0=1$}
\end{align*}

\textbf{Basic calculations}: Assume $\mathbb{E}[\varepsilon_t]=0$. Then: 
\begin{align*}
    \mu_t=\mathbb{E}[\Romanbar{X}_t]=\mathbb{E}[\sum_{i=0}^q \theta_i \varepsilon_{t-1}]=\sum_{i=0}^q \theta_i \mathbb{E}[\varepsilon_{t-1}]=0 \\
\sigma^2_t=var(\Romanbar{X}_t)=\sum_{i=0}^q \theta_i^2 var(\varepsilon_{t-1}) 
\end{align*}

\textbf{Invertibility of Moving Averages}
\begin{align*}
    \Romanbar{X}_t&= \sum_{i=0}^q \theta_i\varepsilon_{t-i}\\
    &=p(B)\varepsilon_t\\
\end{align*}
$p(B)$ is the "Lag polynomial" or "Backshift polynomial"\\

\textbf{Theorem}: $\varepsilon_t$ can be written as $\varepsilon_t=q(B)\Romanbar{X}_t$ for some (possibly infinite) lag polynomial $q(B)$ if all roots of $p$ (i.e., meaning complex solutions to $p(z)=0$) have $|z|>1$, equivalently lie outside the unit circle. \\

\textbf{Motivation}: The imposition of an invertibility condition ensures that there is a unique MA process for a given ac.f.\\ 

\textbf{Example:} In Random Walk (AR process):
\begin{align*}
    \Romanbar{X}_t&=\Romanbar{X}_{t-1}+\varepsilon_t\\
    \Rightarrow (1-B)\Romanbar{X}_t&=\varepsilon_t
\end{align*}
If $p(B) = 1-B$, then:
\begin{align*}
    p(B)\Romanbar{X}_t&=\varepsilon_t
\end{align*}

The roots of $p(B)$ thought of as a regular polynomial $p(z)$ are $p(z)=1-z$, which vanishes at $z=1$.
\begin{itemize}
    \item Called unit root because $|z|=1$
    \item Unit roots $\Rightarrow$ non-Stationarity
\end{itemize}
 Know random walk is not stationary:
\[\sigma^2_t=t \cdot var(\varepsilon_1)\]

\textbf{Note:} Finite-order MA processes are always stationary, but not necessarily invertible. Infinite-order MA processes however, can be non-stationary (see. exc. 3.3)

\subsection{Autoregressive Process}

$\Romanbar{X}_t$ is an autoregressive process of order $p$ if:
\[\Romanbar{X}_t=\varphi_0+\varphi_1\Romanbar{X}_{t-1}+\varphi_2\Romanbar{X}_{t-2}+...+\varphi_p\Romanbar{X}_{t-p}+\varepsilon_t\]

This can be written as a Lag polynomial applied to $\Romanbar{X}$:
\begin{align*}
    q(B)\Romanbar{X}_t&=\varepsilon_t+\varphi_0\\
    q(B)&=1-\varphi_1\Romanbar{X}_{t-1}-...-\varphi_p\Romanbar{X}_{t-p}
\end{align*}

\begin{itemize}
    \item AR(1):
    \begin{align*}
        X_t&=\mu_t+\alpha_1 BX_t +\varepsilon_t\\
        &=\mu+\alpha X_{t-1}+\varepsilon_t
    \end{align*}
    \item AR(p): $X_t=\mu+\alpha_1 X_{t-1}+...+\alpha_pX_{t-p}+\varepsilon_t $
    \item[] Regression of $X$ on its own past
\end{itemize}

Any AR($p$) process:
\begin{itemize}
    \item Can be written as an MA($\infty$) process, but there are no easy formulas for the MA coefficients when $p\geq 2$
    \item Is, by definition, invertible.
\end{itemize}

The process is stationary if the roots of the equation \[
\phi(x)=1-\sum_{i=1}^p \alpha_i x^{i} = 0
\] (where $x$ is potentially complex) lie outside the unit circle.\\

To find the ac.f. in the stationary case, from $X_t=\sum_{i=1}^p \alpha_i X_{t-i}+Z_t$, where the $\alpha_i$ are constants, we get:\[\rho(k)=\sum_{i=1}^p \alpha_i\rho(k-i) \quad \forall k>0\]

This set of equations is called the \textbf{Yule-Walker equations}.\\

The set of Yule-Walker equations has the general solution \[
\rho(k) = \sum_{i=1}^p A_i\pi_i^{|k|}
\]
where the $\{\pi_i\}$ are the roots of the so-called auxiliary equation \[
y^p - \sum_{i=1}^p \alpha_i y^{p-i}=0
\]
The constants $\{A_i\}$ are chosen to satisfy the initial conditions:
\begin{itemize}
    \item $\rho(0)=1 \Rightarrow \sum A_i = 1 $
    \item The first ${p-1}$ Yule-Walker equations provide $(p-1)$ further restrictions on the $\{A_i\}$, using $\rho(0)=1$ and $\rho(k)=\rho(-k)$
\end{itemize}
Alternative (but equivalent) condition for stationarity:
\begin{itemize}
    \item The roots $\{\pi_i\}$ of the auxiliary equation all statisfy $|\pi_i|<1$
\end{itemize}


\subsection{ARMA(p,q)}

ARMA combines MA and AR processes:

\begin{itemize}
    \item ARMA(1,1): \[
    X_t = \overbrace{\mu}^{\mathclap{\text{Intercept parameter}}} + 
    \underbrace{\alpha X_{t-1}}_{\mathclap{\text{Autoregressive component}\phantom{ \quad \quad \quad \quad\quad \quad \quad}}} + 
    \overbrace{\varepsilon}^{\mathclap{\text{Shock}}} + 
    \underbrace{\beta \varepsilon_{t-1}}_{\mathclap{\text{Moving average component}}}
    \]
    \item ARMA(p,q): \[
    X_t=\mu+\alpha_1 X_{t-1} +...+\alpha_p X_{t-p}+ \varepsilon_t + \beta_1 \varepsilon_{t-1}+...+ \beta_q \varepsilon_{t-q}
    \]
    \begin{itemize}
        \item "Regression on own history and shock history"
        \item Example: \(X_t=\mu+\Phi(B)X_t+\varepsilon_t+\psi(B)\varepsilon_t \)
    \end{itemize}
\end{itemize}

The condition for \textbf{stationarity} is that the AR part $\phi(B)$ is stationary.\\
The condition for \textbf{invertibility} is that the MA part $\theta(B)$ is invertible.

\subsection{ARIMA(p,d,q)}

ARIMA Model for \(\nabla X_t=X_t-X_{t-1}\)
\begin{itemize}
    \item[] E.g., \( ARIMA(1,1,1)=\nabla X_t=\mu +\alpha \nabla X_{t-1} + \varepsilon_t + \beta \varepsilon_{t-1} \)
    \item $ARIMA(p,d,q)$ \quad $d$: number of times $\nabla$ is applied
\end{itemize}

\subsection{SARIMA}

\[SARIMA(p,d,q) \times (P,D,Q)_s\]

In practice, many time series contain a seasonal periodic component, which repeats every $s$ observations. \\

A general multiplicative seasonal ARIMA model (SARIMA) is defined as: \[
\phi_p(B) \Phi_P(B^s)W_t = \theta_q(B)\Theta_Q(B^s)Z_t 
\]
where $B$ denotes the backwards shift operator, $\phi_p, \Phi_P, \theta_q, \Theta_Q$ are polynomials of order $p,P,q,Q$, respectively, $Z_t$ denotes a purely random process and \[ W_t=\nabla^d \nabla^D_s X_t \] denotes the differenced series.

\textbf{Example}: $SARIMA(1,0,0)\times(0,1,1)_{12}$
\begin{align*}
    \phi_1(B) \cancel{\Phi_0(B^{12})} \cancel{\nabla^0} \nabla^1_{12} X_t = \cancel{\theta_0(B)}\Theta_1(B^{12})Z_t \\
    (1-\phi_1 B)(X_t-X_{t-12})=(1+\Theta B^{12})Z_t \\
    X_t = X_{t-12} + \phi_1(X_{t-1}-X_{t-13}) + Z_t + \Theta_1 Z_{t-12}
\end{align*}

\subsection{Exercise 3}
\begin{itemize}
    \item Deriving ac.f. 
    \begin{enumerate}
        \item[] $\rho(k)=\frac{Cov(X_\cdot, X_{c\cdot + k})}{Var(X_\cdot)} $
        \item Calculate $\mathbb{E}[X_t]$
        \item Calculate $\mathbb{E}[X_t X_{t+k}] = Cov$
        \item Calculate $Var(X_t)$
    \end{enumerate}
    \item Assessing Stationarity:
    \begin{itemize}
        \item Look at the characteristic polynomial of an AR process, if all roots lie outside the unit circle ($|root|>1$) the model is stationary.
    \end{itemize}
    \item Assessing Invertibility:
    \begin{itemize}
        \item Look at the characteristic polynomial of an MA process, roots again must lie outside the unit circle
    \end{itemize}
    \item Reminder: quadratic formula:
    \begin{align*}
        \text{If } ax^2 +bx+c=0 \\
        x=\frac{-b \pm \sqrt{b^2-4ac}}{2a}
    \end{align*}
\end{itemize}