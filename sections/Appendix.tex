\subsection{Determinant Calculation} \label{Determinant Calculation }

Calculating the determinant of a matrix is a fundamental concept in linear algebra. The determinant is a scalar value that can be computed from the elements of a square matrix and provides important properties about the matrix, such as whether it is invertible.

\subsection*{Steps to Calculate the Determinant}

\subsubsection*{Determinant of a 2x2 Matrix}
For a $2 \times 2$ matrix:
\[
A = \begin{pmatrix}
a & b \\
c & d
\end{pmatrix}
\]
The determinant, denoted as $\det(A)$ or $|A|$, is calculated as:
\[
\det(A) = ad - bc
\]

\subsubsection*{Determinant of a 3x3 Matrix}
For a $3 \times 3$ matrix:
\[
A = \begin{pmatrix}
a & b & c \\
d & e & f \\
g & h & i
\end{pmatrix}
\]
The determinant is calculated using the rule of Sarrus or cofactor expansion:
\[
\det(A) = a(ei - fh) - b(di - fg) + c(dh - eg)
\]

\subsubsection*{General Case for $n \times n$ Matrix}
For larger matrices, the determinant can be calculated by cofactor expansion along any row or column. The process involves breaking the determinant into smaller determinants of submatrices.
