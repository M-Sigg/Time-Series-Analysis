\subsection{Arithmetic Series}
General Formula: \[
S_n=\frac{n(a_1 + a_n)}{2}
\]

\subsection{Geometric Series}
Given an integer $n_0$ and a real number $0<a \neq 1$:
\begin{align*}
    \sum_{i=0}^n a^{i} = 1 + a + a^2 + \cdots +a^n = \frac{1-a^{n+1}}{1-a} \label{GEOSeries}
\end{align*}
Given an integer $n_0$ and a real number $|a|<1$:
\begin{align*}
    \sum_{i=0}^\infty a^{i} = \frac{1}{1-a}
\end{align*}

\subsection{Determinant Calculation} \label{Determinant Calculation }

Calculating the determinant of a matrix is a fundamental concept in linear algebra. The determinant is a scalar value that can be computed from the elements of a square matrix and provides important properties about the matrix, such as whether it is invertible.

\subsection*{Steps to Calculate the Determinant}

\subsubsection*{Determinant of a 2x2 Matrix}
For a $2 \times 2$ matrix:
\[
A = \begin{pmatrix}
a & b \\
c & d
\end{pmatrix}
\]
The determinant, denoted as $\det(A)$ or $|A|$, is calculated as:
\[
\det(A) = ad - bc
\]

\subsubsection*{Determinant of a 3x3 Matrix}
For a $3 \times 3$ matrix:
\[
A = \begin{pmatrix}
a & b & c \\
d & e & f \\
g & h & i
\end{pmatrix}
\]
The determinant is calculated using the rule of Sarrus or cofactor expansion:
\[
\det(A) = a(ei - fh) - b(di - fg) + c(dh - eg)-ö
\]

\subsubsection*{General Case for $n \times n$ Matrix}
For larger matrices, the determinant can be calculated by cofactor expansion along any row or column. The process involves breaking the determinant into smaller determinants of submatrices.


\subsection{Maximum Likelihood Estimation (MLE)}

Maximum Likelihood Estimation (MLE) is a method for estimating the parameters of a statistical model. The main idea is to find the parameter values that maximize the likelihood function, which measures how well the model explains the observed data.

\subsubsection*{Steps in Maximum Likelihood Estimation}

1. \textbf{Define the Likelihood Function}:
   Given a set of observations \(\{x_1, x_2, \ldots, x_n\}\), and a statistical model with parameters \(\theta\), the likelihood function \(L(\theta)\) is defined as the probability of observing the given data under the model:
   \[
   L(\theta) = P(X_1 = x_1, X_2 = x_2, \ldots, X_n = x_n \mid \theta)
   \]
   For independent observations, the likelihood function can be written as:
   \[
   L(\theta) = \prod_{i=1}^n P(X_i = x_i \mid \theta)
   \]

2. \textbf{Log-Likelihood Function}:
   It is often more convenient to work with the log-likelihood function, which is the natural logarithm of the likelihood function. The log-likelihood function \(\ell(\theta)\) is given by:
   \[
   \ell(\theta) = \log L(\theta) = \sum_{i=1}^n \log P(X_i = x_i \mid \theta)
   \]

3. \textbf{Maximize the Log-Likelihood}:
   Find the parameter values \(\hat{\theta}\) that maximize the log-likelihood function. This involves solving the following optimization problem:
   \[
   \hat{\theta} = \arg\max_\theta \ell(\theta)
   \]
   This can be done using calculus by setting the derivative of the log-likelihood function with respect to \(\theta\) to zero and solving for \(\theta\):
   \[
   \frac{\partial \ell(\theta)}{\partial \theta} = 0
   \]

4. \textbf{Second Derivative Test}:
   To ensure that the solution obtained is a maximum, the second derivative of the log-likelihood function can be evaluated. If the second derivative is negative, it indicates a maximum:
   \[
   \frac{\partial^2 \ell(\theta)}{\partial \theta^2} < 0
   \]

\subsubsection*{Example}

Consider a simple example where we assume the data \(\{x_1, x_2, \ldots, x_n\}\) are drawn from a normal distribution with unknown mean \(\mu\) and known variance \(\sigma^2\).

1. \textbf{Likelihood Function}:
   \[
   L(\mu) = \prod_{i=1}^n \frac{1}{\sqrt{2\pi\sigma^2}} \exp\left(-\frac{(x_i - \mu)^2}{2\sigma^2}\right)
   \]

2. \textbf{Log-Likelihood Function}:
   \[
   \ell(\mu) = \sum_{i=1}^n \left( -\frac{1}{2} \log(2\pi\sigma^2) - \frac{(x_i - \mu)^2}{2\sigma^2} \right)
   \]

3. \textbf{Maximize the Log-Likelihood}:
   Taking the derivative with respect to \(\mu\) and setting it to zero:
   \[
   \frac{\partial \ell(\mu)}{\partial \mu} = \sum_{i=1}^n \frac{x_i - \mu}{\sigma^2} = 0
   \]
   Solving for \(\mu\):
   \[
   \hat{\mu} = \frac{1}{n} \sum_{i=1}^n x_i
   \]
   This is the sample mean, which is the MLE for the mean of a normal distribution with known variance.

\subsubsection*{Advantages and Disadvantages}

\textbf{Advantages}:
\begin{itemize}
    \item Provides a unified framework for parameter estimation.
    \item Asymptotically efficient: MLE estimators achieve the lowest possible variance among unbiased estimators for large sample sizes.
\end{itemize}

\textbf{Disadvantages}:
\begin{itemize}
    \item Requires the specification of the likelihood function, which might be complex for some models.
    \item Can be computationally intensive, especially for large datasets or complex models.
    \item Sensitive to model assumptions; incorrect assumptions can lead to biased estimates.
\end{itemize}
